% #######################################
% ########### FILL THESE IN #############
% #######################################
\def\mytitle{Karnaugh-map Using Arduino}
\def\mykeywords{}
\def\myauthor{M.DINESH}
\def\contact{maddudinesh12@gmail.com}
\def\mymodule{FUTURE WIRELESS COMMUNICATIONS-(FWC22044)}
% #######################################
% #### YOU DON'T NEED TO TOUCH BELOW ####
% #######################################
\documentclass[10pt, a4paper]{article}
\usepackage[a4paper,outer=1.5cm,inner=1.5cm,top=1.75cm,bottom=1.5cm]{geometry}
\twocolumn
\usepackage{graphicx}
\graphicspath{{./images/}}
%colour our links, remove weird boxes
\usepackage[colorlinks,linkcolor={black},citecolor={blue!80!black},urlcolor={blue!80!black}]{hyperref}
%Stop indentation on new paragraphs
\usepackage[parfill]{parskip}
%% Arial-like font
\usepackage{lmodern}
\renewcommand*\familydefault{\sfdefault}
%Napier logo top right
\usepackage{watermark}
%Lorem Ipusm dolor please don't leave any in you final report ;)
\usepackage{karnaugh-map}
\usepackage{enumerate}
\usepackage{tabularx}
\usepackage{lipsum}
\usepackage{xcolor}
\usepackage{listings}
%give us the Capital H that we all know and love
\usepackage{float}
%tone down the line spacing after section titles
\usepackage{titlesec}
%Cool maths printing
\usepackage{amsmath}
%PseudoCode
\usepackage{algorithm2e}

\titlespacing{\subsection}{0pt}{\parskip}{-3pt}
\titlespacing{\subsubsection}{0pt}{\parskip}{-\parskip}
\titlespacing{\paragraph}{0pt}{\parskip}{\parskip}
\newcommand{\figuremacro}[5]{
    \begin{figure}[#1]
        \centering
        \includegraphics[width=#5\columnwidth]{#2}
        \caption[#3]{\textbf{#3}#4}
        \label{fig:#2}
    \end{figure}
}

\lstset{
frame=single, 
breaklines=true,
columns=fullflexible
}

\title{\mytitle}
\author{\myauthor\hspace{1em}\\\contact\\IITH\hspace{0.5em}-\hspace{0.5em}\mymodule}
\date{}
\hypersetup{pdfauthor=\myauthor,pdftitle=\mytitle,pdfkeywords=\mykeywords}
\sloppy
% #######################################
% ########### START FROM HERE ###########
% #######################################
\begin{document}
 \maketitle
 \tableofcontents
 \begin{abstract}
     
  The objective of this manual is to show how to \\verify following min-terms.F= (m7+m2+m6+m5) \\using karnaugh-map
 \end{abstract}
    
 

 \section{Introduction}
    Karnaugh-map provides a systematic method for
    \\simplifying boolean expressions and may produce
    \\simplest SOP or POS expressions.
    
    
    karnaugh-map used to minimize number of logic
    \\gates that are required in a digital circuit.
    
    
    
    
    
    \section{components}
    \begin{tabularx}{0.4\textwidth} { 
  | >{\centering\arraybackslash}X 
  | >{\centering\arraybackslash}X 
  | >{\centering\arraybackslash}X 
  | >{\centering\arraybackslash}X | }
  \hline
  component & value & quantity \\
  \hline
  Arduino & UNO & 1 \\
  \hline
  Breadboard & - & 1 \\ 
  \hline
  Led & - & 1 
   \\
  \hline
  Resistor & 220ohm & 1 \\
  \hline
  Jumperwires & M-M & 10 \\
  \hline
  \end{tabularx}
  \begin{center}
      Table-0
  \end{center}
  
  
  

    
    
 
 
 \section{karnaugh-map}
 \subsection{Implementation}
 \begin{karnaugh-map}[4][4][1][$CD$][$AB$]
    \minterms{2,5,6,7}
    \maxterms{0,1,3,4,8,9,10,11,12,13,14,15}
    \implicant{2}{6}
    \implicant{5}{7}
    \implicant{7}{6}
    \end{karnaugh-map} 
    \\ 
    \begin{center}
        Figure 1:k-map
    \end{center}
       
        From the above karnaugh-map the expression is
       
       
       A'BD+A'BC+CD'A'
       
       This karnaugh-map is verified by using 
       
       
       Truthtable Table-1
       
     \section{Truthtable}
  c\begin{tabularx}{0.4\textwidth} { 
  | >{\centering\arraybackslash}X 
  | >{\centering\arraybackslash}X 
  | >{\centering\arraybackslash}X 
  | >{\centering\arraybackslash}X
  | >{\centering\arraybackslash}X |}
  \hline
  A & B & C & D & O/P \\
  \hline
  0 & 0 & 0 & 0 & 0  \\
  \hline
  0 & 0 & 0 & 1 & 0  \\
   \hline
  0 & 0 & 1 & 0 & 1  \\
   \hline
0 & 0 & 1 & 1 & 0  \\
   \hline
  0 & 1 & 0 & 0 & 0  \\
   \hline
  0 & 1 & 0 & 1 & 1  \\
   \hline
  0 & 1 & 1 & 0 & 1  \\
   \hline
  0 & 1 & 1 & 1 & 1  \\
   \hline
  1 & 0 & 0 & 0 & 0  \\
   \hline
  1 & 0 & 0 & 1 & 0 \\
   \hline
  1 & 0 & 1 & 0 & 0  \\
   \hline
  1 & 0 & 1 & 1 & 0  \\
   \hline
  1 & 1 & 0 & 0 & 0  \\
   \hline
  1 & 1 & 0 & 1 & 0  \\
   \hline
  1 & 1 & 1 & 0 & 0  \\
   \hline
  1 & 1 & 1 & 1 & 0  \\
  \hline
  
  \end{tabularx}
  \begin{center}
      Table-1
  \end{center}
\section{Hardware Connections}
1.connect the arduino to the computer



\begin{tabularx}{0.5\textwidth} { 
  | >{\centering\arraybackslash}X 
  | >{\centering\arraybackslash}X 
  | >{\centering\arraybackslash}X 
  | >{\centering\arraybackslash}X
  | >{\centering\arraybackslash}X 
  | >{\centering\arraybackslash}X
  | >{\centering\arraybackslash}X |}
  \hline 
  arduino & 2 & 3 & 4 & 5 & 9  & gnd  \\
  \hline
  input &A & B & C & D &  & \\
  \hline
  led  &  &  &  &  & + & -  \\
  \hline
  \end{tabularx}
  
2.The led will ON and OFF when changing the inputs 
  

  \begin{center}
      Table-2
  \end{center}



       
\section{Software}
  Make the connections and connect the Vaman Board to the PC via USB.In the location of choice,type  the below commands
  \begin{enumerate}[1.]
	  \item $svn co https://github.com/maddudinesh/iithyderabad-fwc/blob/main/arm_examples/arm_assignment$
\item cd $flash/GCC_Project/$
\item make
\item cd ../../
\item bash $scp_send.sh flash$
\end{enumerate}
  

  
       
  
\end{document}
